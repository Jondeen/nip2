\chapter{Configuration}
\mylabel{sec:config}

Click on the \ctr{Edit}\ct{Preferences} to see all the preference
options. There are a lot of things you can change (probably too many). This
section will list the most important.

\subsection{Calculation}

This column has the options which control how \nip{} starts and how and when
it calculates.

\begin{description}

\item[\ct{Data path}]

This is a list of directories where \nip{} searches for data files. These are
any files that \nip{} can use but which aren't loaded at startup. I usually
append the main areas on my machine where I store image files, for example.

The default value is \ct{["\$HOME/.\$PACKAGE-\$VERSION/data",
"\$VIPSHOME/share/nip/data", "."]}.

\item[\ct{Temporary files}] 

This is where any intermediate files will be stored.  It defaults to a
directory called \ct{tmp} under your home area's \ct{.nip2-xx} directory. If
\ct{nip} crashes, it may leave old files here.

\item[\ct{Start path}]

This lists directories which are searched when \nip{} starts for any loadable
files. Anything that \nip{} comes across will be loaded up. 

The default value is \ct{["\$HOME/.\$PACKAGE-\$VERSION/start",
"\$VIPSHOME/share/nip/start"]}.

\item[\ct{Auto-recalc}] 

With this on (the default) \nip{} will recalculate whenever anything changes.
Turn this off if recalculations are taking a long time and you want to make a
series of small changes.

\item[\ct{Update sliders during drag}] 

This sets whether recalculation happens as sliders are dragged, or whether the
recalculation waits until the drag finishes. There's a similar setting for
regions.

\item[\ct{Auto workspace save}]

With this tured on (the default) \nip{} will save the current workspace to
the temporary files area a second after the last recalculation. If \nip{}
crashes, you can restart it and click \ctr{File}\ct{Search for Workspace
Backups} and \nip{} will reload the last workspace where you made a change.

\item[\ct{Auto-reload on file change}]

With this turned on \nip{} will automatically reload any image files that
change while it has them open. Handy if you're using \nip{} to watch a file
that another program is updating.

\item[\ct{Maximum text display}]

This sets the number of characters \nip{} shows for string values. Turn it up
if you want to see inside long strings.

\item[\ct{Maximum heap}]

This sets the limit on the heap size. Turn it up if you start getting \ct{Heap
full} error messages. If you left-click on the space free label in the bottom
right of the main window, it will change to display the current heap
statistics. There's a useful tooltip as well.

\item[\ct{Number of CPUs to use}]

If you have a machine with more than one CPU, you can make \nip{} faster by
upping this number.

\end{description}

\subsection{Image display}

This set of options control the default image display window settings. Useful
if you're always having to turn the status bar on (for example). The maximum
size option is handy if you're using \nip{} on a machine with a small display.

The \ct{Auto popup} option makes \nip{} pop up an image display window 
automatically whenever you make a new image object.

\subsection{Other options}

Other areas of preferences are less useful. 

\begin{description}

\item[\ct{Display LEDs}]

If you're using a theme which uses bitmaps for widgets, you won't be able to
see the button colour changes \nip{} usually uses to indicate state. This
option adds three small LEDs to each row which indicate select, busy and
error.

\item[\ct{Default image format}]

By default \nip{} file browsers show only VIPS images. If you find you mostly
use (for example) JPEG images, you'll save yourself a few clicks on every file
operation by switching this option to JPEG format.

\item[Image format options]

You can set the save options for the various image formats. 

\item[Video for linux]

If you running Linux and have a capture card that supports the V4L interface,
you can capture straight from the card into \nip{}. Set the capture options
here.

\item[Paintbox]

The paintbox normally tracks all undo operations. This can chew up a lot of
memory, especially for flood fills. Reduce the number of undo steps to free 
up some RAM.

\end{description}

