\chapter{Reference}
\mylabel{sec:reference}

\noindent
This chapter is supposed to be a user-interface reference.
\cref{sec:menus} describes the items in the \ct{Toolkits} menu
and \cref{sec:program} is the programming language 
reference. \cref{sec:tutorial} has a tutorial-style introduction.

\section{Image view window}
\mylabel{sec:view}

\fref{fg:imageview} shows \nip{}'s image view window with all the toolbars
turned on.

If you press \ct{i} (or \ct{+}) with the keyboard focus on the image
you will zoom in on the pixel your mouse pointer is over. Press \ct{o}
(or \ct{-}) to zoom out again, or press the number keys \ct{1}, \ct{2},
\ct{4} and \ct{8} to jump straight to a particular magnification. If you
hold down the Ctrl key while pressing these numbers, \nip{} will zoom
out by that amount. If you press \ct{0} (the number zero), then \nip{}
will pick a magnification or reduction which fits the image to the size
of the window.

When the image is too large for the window, you can use the scroll bars
to move about the image. With the keyboard focus on the image the cursor
keys left, right, up and down move a few pixels in each direction; hold down
Shift as well to move a screenful at a time; hold down Ctrl as well to jump
to the extreme edges of the image. 

You can also drag with the middle mouse button to pan around the image. Use
the mousewheel to pan up and down, hold down Shift and the mousewheel to
pan left and right.  Use Ctrl and the mousewheel to zoom in and out.

Use the \ct{View} menu to add extra elements to the window.  You can turn
the status bar on and off, and you can add a display control bar, a paintbox
and a set of rulers to the window. 

\begin{fig2}
\figw{6in}{ir2.jpg}
\caption{The display control bar}
\mylabel{fg:scr3}
\end{fig2}

If you select \ctr{View}\ctr{Toolbar}\ct{Display Control}, \nip{} will add a
bar to the top of the window which you can use to change the contrast and
brightness of the image you are viewing. The left-hand slider and text box
set the gain for the image: each pixel is multiplied by this amount
before display. The right-hand slider and text box set the offset:
each pixel has this value added to it before display. This is useful
for boosting the brightness in dark areas of images.

\begin{figure}
\figw{1.5in}{ir3.jpg}
\caption{The display control bar menu}
\mylabel{fg:scr4}
\end{figure}

If you click the left mouse button on the arrow to the left of the display 
control bar, \nip{} pops up a menu of useful display functions ---
see \fref{fg:scr4}.

\ct{Scale} searches the area of the image you are viewing for the darkest
and brightest points and chooses settings for the gain and offset sliders
which will stretch the image to use the full range of your screen. \ct{False
colour} tries to make small differences in brightness more visible by
colour-coding them.

If \ct{Interpret} is turned on (it is by default), then \nip{} will look
at the \ct{Type} field in the image header, and use that as a hint when
transforming the image to a viewable form for you. This is usually the
behaviour you want.  \ct{Reset} moves the sliders back to the default
positions of 1.0 and 0.0. \ct{Set As Workspace Default} makes the current
display bar settings the default for all new image windows in this workspace.
Finally, \ct{Hide} removes this display control
bar.

If you select \ctr{View}\ctr{Toolbar}\ct{Rulers}, \nip{} will add rulers to the
edges of the window which you can use to measure numbers of pixels.
If you left-drag from the ruler, you can create a guide.
Guides are useful for lining up other things in the view window, and also
affect paint box actions. A right-button menu on the rulers lets you use a mm
scale rather than a pixel scale, and controls whether the \ct{Xoffset} and
\ct{Yoffset} header fields are used.

The \ct{File} menu contains two useful items: select \ct{Replace
Image} to change the image which is being displayed in the window (you can
also drag and drop new images in). Select
\ct{Save Image} to save the image you are viewing to a file. See
\pref{sec:loadsave} for details on \nip{}'s load and save dialogs. The
\ct{New} menu is a no-mouse route for creating regions, points, guides
and arrows.

\mylabel{sec:paintbox} 
If you select \ctr{View}\ctr{Toolbar}\ct{Paint}, \nip{} adds a paint bar
to the top of the window. You can use the paint bar to do simple edits to
the image being displayed. See \fref{fg:paint}.

While the paint bar is very limited, it does have two useful features. First,
it can paint with any pixel value, even complex. For example you can take the
fourier transform of an image and paint out the peaks. Secondly, it doesn't
operate on a memory copy of an image, it operates directly on the file on
disc. This means that you can paint on images of any size, but it does make
the paint bar a bit dangerous.

Normally paint actions are live, that is, every time you paint something all
the objects which depend on the thing you painted will recalculate. This can
sometimes cause annoying delays: there's a preferences option to turn off
automatic recalculations for the paint bar.

\begin{fig2}
\figw{6in}{ir4.jpg}
\caption{The paint bar}
\mylabel{fg:paint}
\end{fig2}

The \ct{Undo} and \ct{Redo} buttons move forward and back though paint
actions. The \ct{Clear} button wipes the undo/redo history (useful if memory
is getting low). There's an option in the preferences workspace which controls
the number of undo steps \nip{} tracks.

You can mark regions on images by holding down Ctrl and dragging down and
right with the left mouse button.  You can move the region about by dragging
on the label with the left mouse button; you can resize it by dragging with
the left mouse button in the border; you can get a useful context menu by
right-clicking on the label; and you can pop up a box which will let you
edit the region numerically by double-left-clicking on the label.  

If you drag up and left, you will make an \emph{arrow}.  If you hold down Ctrl
and just click the left mouse button, you will make a \emph{point}. If you
drag from a horizontal or vertical ruler, you'll make a \emph{guide}. Guides
are useful for lining up other things in the view window.

Use \ctr{File}\ct{New} to make regions, points, arrows and guides without
the mouse.

\section{File select dialogs}
\mylabel{sec:loadsave}

On most platforms you can drag files from your file manager directly to
\nip{}'s main window.  Alternatively, if you select \ctr{File}\ct{Open}
in the main window, \nip{} will pop up a file dialog, see
\fref{fg:open}. The open dialog has the following extra features:

\begin{figure}
\figw{2in}{ir5.jpg}
\caption{Open dialog}
\mylabel{fg:open}
\end{figure}

\begin{description}

\item[Pin up button]
Normally the file dialog closes after you have opened something.
If this item is checked, the dialog will stay up instead --- this is useful
if you want to load or save a series of objects.

\item[Image type select]
Use this menu to select the type of file you want \nip{} to display.
There's a preference option to set the default image format.  The VIPS
file format is fast and accurate, but sadly not very widely supported
(joke). You can also load and save images in TIFF, JPEG, PNG, HDR, CSV and
PBM/PGM/PPM formats. You can usually load in many more formats, it depends
how your \nip{} has been configured.

\item[Show thumbnails]
Pressing this button pops up a window which shows thumbnail-sized images
for all the matching files in the current directory.

\end{description}

The save dialog is a little different, see 
\fref{fg:save}. 

\begin{figure}
\figw{2in}{ir6.jpg}
\caption{Save dialog}
\mylabel{fg:save}
\end{figure}

If pin up and increment are both selected, then after a save \nip{} will
attempt to add one to the selected file name. For example, if you save a file
called \ct{fred001.v}, after the save \nip{} will put the name \ct{fred002.v}
into the selected file name box. Again, this is useful if you want to save
a series of images.

\section{Image processing window}
\mylabel{sec:ipwindow}

\fref{fg:startup} show's \nip{}'s main image processing window. The centre
area is the workspace, the left-hand area is a pane you can reveal to write
custom definitions for this workspace (see \ctr{View}\ct{Workspace
Definitions}), and the right-hand pane is the toolkit browser (see
\ctr{View}\ct{Toolkit Browser}).

Drag with the middle mouse button to scroll the workspace window. Drop a file
on to the workspace background (from your file manager) to load that file.
If you right-click on the workspace background, a useful menu will appear.

\begin{figure}
\figw{3in}{ir7.jpg}
\caption{\nip{}'s main image processing window}
\mylabel{fg:startup}
\end{figure}

\begin{description}

\item[Workspace]
This area displays the objects (images, numbers and so on) which
are currently loaded into \nip{}. The workspace is divided into columns of
objects which each behave rather like windows: they can be moved around,
folded away, loaded, saved and deleted. 

\item[Current column]
One column is the current column. This is the column to
which all new objects are added. Single-left-clicking on the title bar of
a column makes that the current column. See \pref{sec:column}.

\item[File, Edit, View]
Use the \ct{File} menu to create or save work\-spaces, to open workspaces
or load other objects into this workspace, to merge workspaces and to search
for workspace backups.  Use the \ct{Edit} menu to select, group, delete and
duplicate sets of objects.  Use \ct{View} to show and hide elements of the
main window, and to set the object view mode.

\item[Toolkits]
This menu contains all of the image processing functions which are currently
loaded into \nip{}. They are generally grouped by object type: all of the
operations on matricies are under \ctr{Toolkits}\ct{Matrix}, for example.

If you select one of these image processing operations, \nip{} will apply
that operation to the bottom few items in the current column (however many are
necessary --- two items for \ctr{Math}\ctr{Arithmetic}\ct{Add}, for example),
or alternatively, if you have selected some objects explicitly, it will try
to apply the operation to the selected objects. See \pref{sec:apply}. As
you move the mouse pointer over menu items \nip{} tries to display some
helpful information about the operation, including the number and type of
arguments the operation expects.

\item[Toolkit Browser]
This side panel shows all the image processing operations again, but this time
as a large flat list you can easily browse. Type into the search box at the
top to filter operations by keyword. Doubleclick on an item to activate it.

\item[Workspace Definitions]
This side pane shows private definitions for this workspace. Programs you
write here are loaded and saved with this workspace. See the Programming
chapter for details on \nip{}'s programming language.

\item[Free space]
This displays the amount of disc space you have left in your temporary
file area. See \pref{sec:config} if you want to change the directory \nip{}
uses to store temporary files.

If you left-click on the label, it changes to display the space \nip{} has
free internally for performing calculations. You can change this limit in the
\ct{Preferences} workspace. Click again to switch back to disc free.

If you have objects selected, this area changes to show the
names of the selected objects.

\item[Status bar]
As you move the mouse pointer about the window, this bar tries to display
useful information about the thing you are pointing at. 

\end{description}

\subsection{Columns}
\mylabel{sec:column}

Columns are split into a number of areas:

\begin{description}

\item[Column name]
Each column has a name. You can pick any name you like when you make a new
column with \ctr{File}\ctr{New}\ct{Column}. There's no way to rename a column,
unfortunately. Objects in the column are named using the column name,
plus a number.

\item[Column title bar]
Drag with the left mouse button held down on the column title bar to move
the column around the workspace. Double-left-click on the title bar to
change the comment attached to the column. Hold down the right mouse button
on the column title bar to pop up a useful menu. 

The items in the menu let you edit the caption, select all the objects
in the column, make a new column which is a copy of this column, save the
column to a file, convert the column into a menu item
(see \pref{sec:diaref}) and remove the whole column.

\item[Column fold button]
Left-clicking on the fold button folds the column away. Use this to hide
columns which you still need, but which you are not interested in just now.

\item[Expression entry]
You can perform calculations by typing expressions directly into this box. For
example, try entering the following expressions, and pressing Return:

\begin{verbatim}
2 + 2
A1 + 120
"My cat likes\nlasagne"
fred = 12
\end{verbatim}

\noindent
The last example shows custom button name creation. Normaly \nip{} will pick a
name for you, but you can chose your own.

\end{description}

\subsection{Rows}
\mylabel{sec:row}

A column holds a number of rows. Each row comes in four main parts, not all of
which are visible for all row values. Rows which represent classes have a pair
or up/down arrows to the left of the row name button which you can use to
control which parts of the row are visible.

\begin{figure}
\figw{1.5in}{ir8a.jpg}
\caption{Components of a workspace row}
\mylabel{fg:row}
\end{figure}

\begin{description}

\item[Row name button]
Each row has a name. The name is normally formed from the name of the current
column, plus a number.

If you double-left-click on the row name button, \nip{} will pop up a viewer
or dialog box for the value of the row. If you left-click, \nip{} will select
that row and deselect all other rows. If you click on an empty space
in the workspace, it will deselect all rows. If you
Ctrl-left-click, \nip{} will toggle selection of that row.  
If you select one row and then Shift-left-click on
another row in the same column it will select the second row and
all the rows in between. If you drag with the left button, you can change
the order of rows in a column.  Hold down the right mouse button for a useful
menu. If you let the mouse linger over a button, a useful tooltip will appear.

\item[Graphic]
If the row's value is a class, and if the class is an instance of one of
\nip{}'s graphic classes, then \nip{} will draw a graphic representation of
the row's value. See \pref{sec:workspaces} for a more detailed explanation.

\item[Members]
If the row has a class for a value, then \nip{} will draw a sub-column listing
the class members. Subcolumn members are in turn rows themselves.

\item[Text]
Finally, the text part normally shows a text representation of the row's
value. If you left-click on the value, it changes to show the formula which
generated that value. You can edit the formula and press Return to change it.

Alternatively, selecting \ctr{View}\ct{Show Formula} toggles 
between displaying values for objects and displaying the formula.

\end{description}

\subsubsection{Object name colours}

\nip{} changes the background colour of the row name button to show the state
of the row. If background colours are not visible (perhaps your theme turns
them off), try turning on the \ct{Display LEDs in workspace} option in
\ct{Preferences}.

Green means the row is selected (click on the background to unselect), red
indicates an error (right-click on the row button ans select \ct{Recalculate}
to see the full text of the error), brown indicates that the row value is out 
of date and needs recalculating and the various blues indicate parent and
child relationships.

\subsection{Applying operations to objects}
\mylabel{sec:apply}

There are three ways you can apply image processing operations to objects
in your workspace:

\begin{enumerate}

\item
Select the object you want to apply the operation to by single-left-clicking
on the object name. When you single-click, the object name will change
colour to show that it is selected, and \nip{} will display the name of
the selected object at the left end of the status bar (this is
useful if the selected object is scrolled off the edge of the window).

You can select additional objects with Ctrl-left-click and
Shift-left-click. This is necessary if you want to use an image
processing operation that takes more than one argument.

Once you have selected the rows (sometimes you need to select them in
a certain order), click on the processing operation you want from the
\ct{Toolkits} menu.

\item
If there are no objects selected when you click on an image processing
operation, \nip{} uses the bottom few items (as many as are needed by the
operation) in the current column. 

\item
You can also type your formula directly into the expresion
entry line at the bottom of the selected column. \cref{sec:program} describes
the syntax in detail, but it's approximately C.

\end{enumerate}

\subsection{Batch processing}
\mylabel{sec:batch}

If you select a number of rows and then click \ctr{Edit}\ct{Group}, \nip{}
will group the rows together. Now if you select the group and click on an item
in the \ct{Toolkits} menu, \nip{} will apply that operation to every item in
the group. You can group groups, and you can mix grouped and non-grouped rows
freely.

If you save a group, \nip{} will write each item in the group to a separate
file, incrementing the filename each time. 

\subsection{Error handling}
\mylabel{sec:error}

If an object in your workspace has an error (for example, if you are trying
to join two images of different types), then the object name button will
turn red to show that this object contains an error and the tooltip for the
button will show the error message.

\subsection{Making menu items out of columns}
\mylabel{sec:diaref}

If you make a column that does
something useful, you can make it into a menu item by following these steps:

\begin{enumerate}

\item
Make your column look nice. Drag with the left mouse button on the object
name buttons to re-order items in the column, and add comments to explain
what are the input fields and what are the output. Double-click on the column
title bar to add a helpful title to the column.

Add a comment by typing your text (enclosed in double quotes) into the line
at the bottom of the column. Left-drag the row to the right place.

\item
Select \ct{Make Column Into Menu Item} from the 
column title-bar menu, see \pref{sec:column}.

This will open up a new dialog box which you can use to set a name for
your new menu item and the name of the top level menu 
the item should be added to. 

\item
That's it. You'll be prompted to save your new toolkit when you try to
quit \nip{}. We recommend you just say \ct{OK} to the suggested location
for the file. Edit your menus with the programming window, see
\pref{sec:progwin}.

\end{enumerate}

\section{The programming window}
\mylabel{sec:progwin}

To pop up the programming window, click on \ctr{Toolkits}\ct{Edit Toolkits} in
\nip{}'s main image processing window. The window shown in \fref{fg:Fred}
should appear.

Each of the things down the left of the program window 
is a toolkit. Each toolkit is a text
file containing a set of definitions in \nip{}'s programming language.
See \cref{sec:program} for details on the language.

If you open a toolkit, \nip{} shows all of the definitions in that file.
If you click on one of these
\nip{} shows the source for that definition in the main part of the
program window.  After editing a definition, click on \ctr{File}\ct{Process} 
to make \nip{} read what you typed, compile it, and update itself.

Click on \ctr{File}\ctr{New}\ct{Tool} to add a new definition to a toolkit,
click on \ctr{File}\ctr{New}\ct{Toolkit} to make a completely new toolkit. You
can also right-click on tools and toolkits to get a context menu, and you
can left-drag tools to move them around within a toolkit or between toolkits.

Some toolkits are loaded from files when \nip{} starts up, others are built
from the VIPS operation database (for example, \ct{\_arithmetic}), and one 
(called \ct{\_builtin}) contains the functions that are built into \nip{}. If
you select a tool and then click on \ctr{Help}\ct{Help on 
Tool}, \nip{} will try to display the relevant section from the VIPS manual
in your web browser. Currently, this works only for things in the VIPS
operation database: try \ctr{\_arithmetic}\ct{im\_add}, for example.
There's a section in the \ct{Preferences} workspace to control which web
browser \nip{} uses and how it asks for a page.

Toolkits and tools whose names begin with an underscore character are not
displayed in the main \ct{Toolkits} menu.  The idea is that they represent
little utility functions, rather than stuff a user might be interested in. See
\pref{sec:tools} for more information on how tools and toolkits are displayed.

You can have several programming windows open at the same time (often useful,
if confusing). The \ct{Edit} menu lets you search for patterns across all
definitions. The \ct{Jump To Definition} item jumps to the definition of a
symbol. 

\mylabel{sec:trace}
The \ct{Debug} menu has items which open a trace window (use this to track
the actions taken by \nip{}'s reduction engine) and which report on unresolved
symbols and list all current errors.

\section{Command-line interface}
\mylabel{sec:cmdline}

You can use \nip{} from the command-line as well as from the GUI. This can be
handy for automation: you can build a workspace and then run it over a whole set
of images, or use \nip{} as part of a larger system. We've make
websites which use \nip{} as the back-end.

In command-line mode \nip{} runs without a GUI of any sort, it doesn't
even need a window system to be installed on the machine. This makes it
possible to use it in a server or batch context.

These notes are for the Unix command-line, but they should work for Windows as
well.

\nip{} has three main modes of operation from the command-line:

\begin{description}

\item[\nip{} \emph{filename1} \ldots{}]
Start \nip{} in GUI mode, loading the command-line arguments as files.
Filenames can be images, workspaces, matricies, toolkits, and so on.

\item[\nip{} \ct{-e} \emph{expression} \emph{arg1} \ldots{}]
Start in no-GUI mode, print the value of \emph{expression}. The list
\ct{argv} is set to be \verb+["filename","arg1",..]+.

\item[\nip{} \ct{-s} \emph{filename} \emph{arg1} \ldots{}] Start
in no-GUI mode, read in \emph{filename} as a set of definitions,
print the value of symbol \ct{main}.  The list \ct{argv} is set to be
\verb+["filename","arg1",..]+.

\end{description}

You can use the \ct{-o} option to send output somewhere other than the
screen. If these modes don't do quite what you need, you can get finer control
of how \nip{} behaves with a set of other options: see the man page for
details.

\subsection{Using expression mode}

The \ct{-e} option is very easy to use. For example:

\begin{verbatim}
nip2 -e "2 + 2"
\end{verbatim}

\noindent
Prints 4 to stdout.

\begin{verbatim}
nip2 -e "99 + Image_file argv?1" -o result.png fred.jpg
\end{verbatim}

\noindent
Loads argv1 (\ct{fred.jpg}), adds 99, writes the result to \ct{result.png}.

\begin{verbatim}
nip2 -e "Matrix [[1,2],[4,5]] ** -1" -o poop.mat
\end{verbatim}

\noindent
Invert the 2x2 matrix and write the result to \ct{poop.mat}.

If the result of the expression is a list, each item is printed on a new
line. For example:

\begin{verbatim}
nip2 -e "[1..5]"
\end{verbatim}

\noindent
Will print the numbers 1 to 5, each on a new line.

If you have a list result and you are using \ct{-o} to direct the output to a
file, the filename will be incremented each time you write. For example:

\begin{verbatim}
nip2 -e "map (add (Image_file argv?1)) [10, 20 .. 50]" -o result1.png fred.jpg
\end{verbatim}

\noindent
Will load \ct{fred.jpg}, add 10, 20, 30, 40 and 50, then save those images to
\ct{result1.png} to \ct{result5.png}.

\subsection{Using script mode}

With the \ct{-s} option you can use \nip{} as a Unix script interpreter.

Create a file in your favourite text editor called \ct{brighten} containing:

\begin{verbatim}
#!/usr/bin/nip2 -s

main
  = clip2fmt infile.format (infile * scale), argc == 3
  = error "usage: infile scale -o outfile"
{
  infile = Image_file argv?1;
  scale = parse_float argv?2;
}
\end{verbatim}

\noindent
The first line needs to be the path to \nip{} on your system. Use \ct{which
nip2} to find the path if you don't know it. Mark the file as executable with
\ct{chmod +x brighten}, then use it on one of your image files with:

\begin{verbatim}
brighten fred.jpg 1.5 -o bright_fred.png
\end{verbatim}

\noindent
See \cref{sec:program} for details on the programming language. This program
multiplies each input pixel by the constant, producing a floating point image,
then then clips the result back to the same format as the original image
(usually 8-bit unsigned).

\nip{} takes a while (a few seconds) to start up, so this isn't going to be
appropriate for small images or simple calculations. But for complex
operations, or operations on large images, this mode can be very useful.

\subsection{Using \ct{--set}}

The \ct{--set} option (which can be abbreviated to \ct{-=}) lets you make
changes to a workspace after loading it.  Suppose the workspace \ct{test.ws}
has a row called \ct{A1} with the value 12.  Then entering:

\begin{verbatim}
nip2 test.ws --set Workspaces.test.A1=45
\end{verbatim}

\noindent
Will, as normal, start \nip{} and load \ct{test.ws}. But before the first
recalculation, \nip{} will change the value of \ct{A1} to be 45.  You can
use \ct{--set} to create new symbols as well.

\subsection{Other modes}

A set of sub-options let you mix up other modes yourself. For example, it's
common to want to run a workspace on many files.

Suppose the workspace \ct{process.ws} loads an image in \ct{A1}, performs some
processing and produces a result image \ct{A10}. If you run \nip{} with:

\begin{verbatim}
nip2 -bp \
  -= 'Workspaces.process.A1=Image_file "fred.jpg"' \
  -= main=Workspaces.process.A10 \
  -o fred.jpg process.ws
\end{verbatim}

\noindent
This will start \nip{} in batch (ie. no GUI) mode (the \ct{-b} switch), 
load \ct{process.ws}, change \ct{A1} to load another file, set \ct{main}
to be the value of \ct{A10} and save the value of \ct{A10} to \ct{fred.jpg} 
(the \ct{-p} switch).

