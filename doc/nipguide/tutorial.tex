\chapter{Tutorial}
\mylabel{sec:tutorial}

This chapter runs very quickly through how to use \nip{}'s user-interface. See
\cref{sec:reference} if you want more details on how the different bits work.

\section{Quick interface tour}
\mylabel{sec:quicktour}

Start up \nip{}. You should see something like \fref{fg:introwin} (the
exact look might be different on your system).

The menus at the top of the window are very ordinary, except for the
\ct{Toolkits} menu which contains all of the image processing operations.
The main part of the window shows a workspace. Workspaces are split into tabs
(across the top) and each tab is made of a set of columns. The current column,
where new objects appear, has a dark green bar across the top. 

Click on \ctr{File}\ct{Open} to get a file dialog and load up an
image. \nip{} can load most image formats, try it and see.  Check the
\ct{Pin up} box to have the dialog remain after you press OK. 
You can also drag files from the desktop or from your file manager.

After you've loaded an image, \nip{} should look like \fref{fg:loadedimage}.
Double click on the thumbnail to open an image view window. Alternatively,
select \ct{Edit} from the right-button menu on the thumbnail. See
\fref{fg:imageview}.

\begin{figure}
\figw{3in}{snap2.jpg}
\caption{After loading an image}
\mylabel{fg:loadedimage}
\end{figure}

\begin{figure}
\figw{3in}{snap3.jpg}
\caption{Image view window}
\mylabel{fg:imageview}
\end{figure}

As well as the standard keymappings, \nip{} has extra shortcuts for
navigating images, see \tref{tb:shortcuts}.  Use the \ctr{View}\ct{Toolbar}
menu to turn on other features. You can have a status bar (shows image
properties, mouse position and pixel value), a display control bar (lets
you change scale and offset for display pixels, click on the arrow on the
left for a useful extra menu), a paint bar and some rulers.

\begin{tab2}
\begin{center}
\begin{tabular}{||l|l||}
\hline
Keys in image display widget  		& Action \\
\hline
\ct{i}, \ct{+} 				& Zoom in on mouse pointer \\
\ct{o}, \ct{-} 				& Zoom out \\
Cursor up/down/left/right 		& Scroll a small amount in direction \\
Shift and cursor up/down/left/right 	& Scroll a screenful in direction \\
Ctrl and cursor up/down/left/right 	& Scroll to edge of image \\
Middle mouse drag 			& Pan image \\
Mouse wheel 				& Scroll up/down \\
Shift and mouse wheel			& Scroll left/right \\
Ctrl and mouse wheel 			& Zoom in/out \\
\ct{0} (zero key) 			& Zoom out to fit image to window \\
\ct{1}, \ct{2}, \ct{4}, \ct{8}  	& Set magnification to 1, 2, 4 or 8 \\
Ctrl and \ct{2}, \ct{4}, \ct{8}		& Set zoom out factor to 2, 4 or 8 \\
\hline
\end{tabular}
\end{center}

\caption{\nip{} shortcuts for the image view window}
\mylabel{tb:shortcuts}
\end{tab2}

You can mark things on an image. Hold down Ctrl and drag down and right with
the left mouse button to mark a region. Ctrl-left-click to mark a point. Drag
up and left to mark an arrow (two points connected by a line). Drag from
the rulers to mark guides. Right-click on a label to get a menu which
you can use to remove or edit one of these things. Left drag on a label
to move the things around, left-drag on the edges or corners to resize.
\fref{fg:imageviewregion} shows the same image with stuff marked on it.

\begin{figure}
\figw{3in}{snap4.jpg}
\caption{Image view window with marked regions}
\mylabel{fg:imageviewregion}
\end{figure}

Clean up any messing about and leave two regions on your image. The main
window should now look something like \fref{fg:main2regions}.

\begin{figure}
\figw{3in}{snap5.jpg}
\caption{Main window, two regions marked}
\mylabel{fg:main2regions}
\end{figure}

There are three rows visible here, \ct{A1}, \ct{A4} and \ct{A7}. Each row
has (from left to right) a pair of up/down arrows (these indicate that
the row contains sub-rows: click on the down arrow several times to
open the row up and see inside), the name button (left-click to select,
Shift-left-click to extend-select, Ctrl-left-click to toggle select, click
on the workspace background to unselect everything, left-drag on the name
button to reorder items within the column, right-click to get a context menu) 
and the thumbnail image.

Left-click on the name button of one of these images to select it, and
then click on \ctr{Toolkits}\ctr{Image}\ctr{Transform}\ctr{Rotate}\ct{Free}
(alternatively, if nothing is selected when you click on one of the toolkit
menus, \nip{} will apply the operation to the bottom object in the current
column). A new row will appear representing the rotation operation. Drag
the slider to rotate the image (or type an angle in degrees into the box
to the left of the slider and press Return). Pick an interpolation type from
the option menu and zoom in on some pixels to check the result. 
See \fref{fg:rotate}.

\begin{figure}
\figw{3in}{snap6.jpg}
\caption{Using \ctr{Rotate}\ct{Free}}
\mylabel{fg:rotate}
\end{figure}

The same thing works for image processing operations that take two arguments.
Left-click on one of your original regions, Ctrl-left-click on the rotated
image (the box at the left of the main window status bar says what's selected
and in what order), and click on \ctr{Toolkits}\ctr{Image}\ctr{Join}\ct{Left
to Right}. A new row appears representing the join operation.

Click on \ct{Background Colour}, type in 128 and press Return. Drag the
\ct{shim} slider to 50 (or type 50 into the box just to the left of the
slider and press Return). See \fref{fg:join}.

\begin{figure}
\figw{3in}{snap7.jpg}
\caption{Using \ctr{Join}\ct{Left to Right}}
\mylabel{fg:join}
\end{figure}

The \ct{Toolkits} menu is large and can be slow and annoying to find
things in, so \nip{} has several shortcuts. First, you can tear-off menus
by clicking on the dotted line at the top. If you're going to be using
one of the sub-menus repeatedly this can save a lot of clicking. Next,
you can set keyboard shortcuts for menu items by moving the mouse pointer
over the item and pressing the key combination you want to set. 

Some systems won't let you edit menu shortcuts by default. For example,
on GNOME, you need to enable this in \ctr{System}\ctr{Preferences}\ct{Menus
\& Toolbars}.

Finally, there is a toolkit browser: this shows the same set of items, but in
an easier-to-browse way. Click on \ctr{View}\ct{Toolkit Browser} and the
browse side-panel will appear, see \fref{fg:browse}. It shows all of the items
as a single long list. Type into the search box at the top to only show items
which match. Double-click (or press Return) on an item to activate it. Scroll
to the right to see what arguments the item needs and what menu it appears in.

\begin{figure}
\figw{3in}{snap7a.jpg}
\caption{The toolkit browser}
\mylabel{fg:browse}
\end{figure}

The box at the bottom of each column is for entering new expressions. You
can type stuff here, and \nip{} will make a new row for each item you
enter. Try typing \ct{2 + 2} and pressing Return. The syntax is (almost,
with a few small differences) the same as the C programming language. See
\pref{sec:operators} for a list of the differences. Try
multiplying the joined images by a small amount (eg. type something like
\ct{A9 * 1.2} and press Return). Normally \nip{} will pick names for new
objects for you (like \ct{A1}), but you can set a name yourself if you like.
Try entering \ct{fred = 12}.

Click the down button once on your brightened image and left-click on the area
just below the thumbnail. You should see the stuff you typed to make that row.
You can edit it to be anything else, press Return and \nip{} will recalculate.
Try going back to your original image (the one you loaded from a file),
open an image view window, and try dragging one of the regions. You can
change any of the sliders in the rotate or the join rows as well.  

You can also edit the insides of objects. Click the down button next to
one of your regions until the \ct{width} and \ct{height} rows appear, click on
\ct{width} and type \ct{height * 2}. Now open the image window the region is
defined on and try to resize it: you'll find that the width of the region is
fixed, but that if you change the height, the width changes with it. This is a
very general property of classes in \nip{}: you can use it to join objects
together in complex ways, and to modify the behaviour of interactive objects. 

Right click on a column title bar to get a useful menu. Click on
\ctr{File}\ctr{New}\ct{Column} make another column (handy for organising
a workspace). If you drag from an image thumbnail on to the workspace
background, \nip{} will make a new column for you. You can drop thumbnails
on to other thumbnails to make links.  If \nip{} falls over (I do hope
it doesn't), you can usually get your work back by restarting \nip{} and
clicking on \ctr{File}\ct{Search for Workspace Backups}.  There are a lot
of preferences (perhaps too many), see \aref{sec:config}.

There is a lot of stuff in the \ct{Toolkits} menus, but they do almost
all have tooltips. If you let your mouse hover over a menu item for a
moment you should get some helpful text. The toolkit menu is organised by
object type. If you want to do something to a matrix, look in the
\ctr{Toolkits}\ct{Matrix} menu. The exception is \ctr{Toolkits}\ct{Tasks}
which repeats many of the regular toolkit items, but groups them by typical
tasks instead.

Operations can work on groups as well as on single images, so you can batch
things up. If you save a group of images, \nip{} will number each image
sequentially for you.  You can use \ctr{Edit}\ct{Duplicate} to make copies
of objects. If you select lots of objects and duplicate them, \nip{} will
(fairly intelligently) rename everything for you so it all still works.

\section{\nip{} for reflectogram mosaics}
\mylabel{sec:irtut}

This section quickly builds an infrared reflectogram mosaic using the
sample images that come with \nip{}. See \cref{sec:ir} for detailed
coverage.

Click on \ctr{File}\ct{Open Examples}. 
You should see a directory called \ct{1\_point\_mosaic}. Doubleclick and
you'll see a file called \ct{1pt\_mosaic.ws}. Doubleclick that and you'll load
the workspace for this example.

If you'd rather make the workspace yourself, click on the file type filter and
select \ct{All files}. A set of 8 images should appear.  Click on the first
file, shift-click on the last, and click \ct{Open}. See \fref{fg:loadsamples}.

\begin{figure}
\figw{3in}{snap14.jpg}
\caption{Loading the sample images}
\mylabel{fg:loadsamples}
\end{figure}

The images have been named to match their positions in the mosaic, so for
example \ct{cd2.1.jpg} is the first image in row two.  Open up viewing windows
for the first two images by double clicking on the thumbnails.
Move the two opened images viewers so that they are side by side. Adjust
the zoom (using the \ct{i} and \ct{o} keys) and the pan (by dragging with the
middle mouse button) so that the overlap
area is visible in both images.

Mark a tie-point on each image by Ctrl-left-clicking on a feature you can see
in both images, see \fref{fg:readyjoin}. Move a point after you've marked it
by dragging on the label. You don't need to be exact: \nip{} just uses the
point you select as the start point for a search. It can cope with
misses of up to about 10 pixels. To mosaic the two images together, click
on \ctr{Toolkits}\ctr{Tasks}\ctr{Mosaic}\ctr{One Point}\ct{Left to Right}. 
See \fref{fg:joined}.

\begin{figure}
\figw{3in}{snap15.jpg}
\caption{Ready to join}
\mylabel{fg:readyjoin}
\end{figure}

\begin{figure}
\figw{3in}{snap16.jpg}
\caption{Joined images}
\mylabel{fg:joined}
\end{figure}

Picking items deep in the toolkit menu is fiddly, so \nip{} has several
shortcuts.  First, you can tear off any toolkit menu by clicking on the dotted
line at the top. Secondly,
you can assign any keyboard accelerator to any menu item. Navigate to the
menu item and while it is selected, press the key combination you want to use
as a shortcut (for example, Ctrl-L might be good for \ctr{Mosaic}\ctr{One
Point}\ct{Left to Right}). Now whenever you press Ctrl-L with the keyboard
focus in the main window, you will do a left-right mosaic join. Finally, you
can use the toolkit browser to display a selection of the tools in a pane on
the right-hand side of the main window. Click on \ctr{View}\ct{Browse
Toolkits}, then type ``mosaic'' into the search box at the top. The toolkit
browser will display all items related to mosaicing.

Some systems won't let you edit menu shortcuts by default. For example,
on GNOME, you need to enable this in \ctr{System}\ctr{Preferences}\ct{Menus
\& Toolbars}.

Join the rest of the pairs of sample images together
left-right. \fref{fg:joins} shows how they should fit together. Once you
have made all the rows, join the rows together in turn to make the complete
image using \ctr{Mosaic}\ctr{One Point}\ct{}\ct{Top to Bottom}. 

When you've built the whole thing you'll see that there are differences
in brightness between the tiles that make up your composite image. You can
fix most problems like this automatically by selecting your final mosaiced
image and clicking on \ctr{Mosaic}\ct{Balance}. This operation takes your
mosaic apart, examines the overlap areas for differences in brightness,
calculates a set of adjustment factors to minimise these differences,
and then rebuilds the mosaic.

There can be some problems left even after mosaic balance. Use
\ctr{Mosaic}\ct{Tilt Brightness} to remove any left-right or up-down
graduations in brightness. 

Save your mosaic workspace for future reference by clicking on
\ctr{File}\ct{Save Workspace}. To save just the mosaiced image, right click
on the thumbnail and select \ct{Save As}.

\section{\nip{} for nerds}
\mylabel{sec:nerdtour}

This section sprints through a bit of \nip{} programming, see
\pref{sec:program} for full details and a more formal definition of the
language.

The insides of \nip{} are built with \nip{}'s own programming language. It's a
pure lazy functional language with classes. It's C's expression syntax
(more or less) plus approximately Miranda/Haskell function syntax,
plus some basic class stuff. \nip{}'s main window is a class browser for this
programming language.

Click on \ctr{Toolkits}\ct{Edit Toolkits} in \nip{}'s main window to pop up the
programming window (see \pref{sec:progwin} for details on all the bits in the
window), then in the edit area there type:

\begin{verbatim}
// add two things
 
Fred a b = class {
  sum = a + b;
}
\end{verbatim}

This defines a class called \ct{Fred} whose constructor takes two arguments,
\ct{a} and \ct{b}. There's one member, called \ct{sum}, which is \ct{a}
and \ct{b} added together.

In the program window, click \ctr{File}\ct{Process}. This makes \nip{}
read what you typed, parse it, compile it and update itself. The program
window should now look like \fref{fg:Fred}.

\begin{figure}
\figw{3in}{snap8.jpg}
\caption{Programming \ct{Fred}}
\mylabel{fg:Fred}
\end{figure}

If you look back at the main \nip{} window, a new menu will have appeared
under \ct{Toolkits} called \ct{untitled}. If you click on that, there will
be a menu item called \ct{Fred}. Let your mouse linger, and you'll see a
tooltip too.

In the main window, type \ct{Fred 2 3} into the box at the bottom of the
current column. Press Return and \nip{} will make a \ct{Fred} for you. Click
on the down arrow to the left of your new \ct{Fred} once to see the members
of \ct{Fred} (just \ct{sum} in this case), click again to see the class
parameters too. The main window should look like \fref{fg:mainFred}.

\begin{figure}
\figw{3in}{snap9.jpg}
\caption{Main window \ct{Fred}}
\mylabel{fg:mainFred}
\end{figure}

Click to the right of \ct{b}, type in a new value and press Return. The
\ct{sum} member should update. \nip{} keeps track of dependencies between
rows, but it also tracks dependencies inside rows, both ones that come
from the class, and ones created by any edits you do to the class instance
after creating it. You won't see it in a simple example, but \nip{} also
discovers and tracks dependencies which can arise at run time.  Click on
the text just to the right of the \ct{b} button again, type \ct{a} and
press Return. Now edit \ct{a}: press Return and both \ct{b} and \ct{sum}
will update.

You can use \ct{Fred} to add any two things together. Click on
\ctr{Toolkits}\ctr{Widgets}\ct{Scale} to make a scale widget, press Ctrl-U
(the keyboard shortcut for \ctr{Edit}\ct{Duplicate}) to duplicate it, and
finally click on \ctr{Toolkits}\ctr{untitled}\ct{Fred}. Open up the new
\ct{Fred} and try dragging some of the scales around. The main window will
look like \fref{fg:slideFred}.

\begin{figure}
\figw{3in}{snap10.jpg}
\caption{Scale \ct{Fred}}
\mylabel{fg:slideFred}
\end{figure}

The scales are classes too (instances of \ct{Scale}). You can open them up
and do strange things with them as well. Open up one of the scales you made
(eg. \ct{A2} in \fref{fg:slideFred}) and change the \ct{from} parameter to
be \ct{A3.value}. Now try dragging the sliders again.

Try dragging the \ct{sum} slider. Now go back and drag one of the
original sliders.  You'll see that \ct{sum} no longer updates, it's stuck
at the last position you dragged it to. This is because there are now two
things affecting the value of \ct{sum}: the underlying code (the \ct{a +
b} inside \ct{Fred}), and the position you dragged the slider representing
\ct{sum} to. \nip{} has the rule that graphical edits (dragging the slider)
override code. To make \ct{sum} update again, right click on the \ct{sum}
button and select \ct{Reset} from the pop up menu. Now drag one of
the input sliders again, and \ct{sum} will start updating once more.

Classes can inherit from other classes. Go back to the program window,
click on \ctr{File}\ctr{New}\ct{Tool} to clear the edit window, and type:

\begin{verbatim}
// multiply two things
 
Jim a b = class Fred a b {
  product = a * b;
}
\end{verbatim}

This defines a class called \ct{Jim} which inherits from \ct{Fred}. Click
\ctr{File}\ct{Process}, then back in
the main window, type \ct{Jim 4 5} into the bottom of the column. Click down
once to expose the members (just \ct{product}), click again to expose the
parameters as well (\ct{a} and \ct{b}), and click a third time to expose
the superclass member (which should be an instance of \ct{Fred}). You can
also open up the \ct{super} member and see inside the \ct{Fred} that this
\ct{Jim} is using as its superclass. See \fref{fg:Jim}.

\begin{figure}
\figw{3in}{snap11.jpg}
\caption{Browsing \ct{Jim}}
\mylabel{fg:Jim}
\end{figure}

\nip{} has about 20 different graphical classes like \ct{Scale}. Whenever a
row takes a new value, \nip{} checks to see if that value is an instance of
one of these special classes, and if it is, it will add a graphical element to
the row display which represents that class's value. It builds the graphical
part by looking inside the class for certain members (for example, the scale
graphic looks for members called \ct{from}, \ct{to} and \ct{value}). When
you change the graphic (maybe by dragging the scale), \nip{} rebuilds
the class by looking inside for a edit member (eg. \ct{Scale\_edit})
or if that's not defined, a constructor member (eg. \ct{Scale}).

You can make your own graphic widgets by subclassing \nip{}'s built-in
ones. By selectively overriding default constructors and adding edit members,
you can control how your new widget will behave in expressions, and how it
will behave if it's edited graphically.

Make a new column, load up an image (use \ctr{File}\ct{Open}), open an
image viewer (double-click on the thumbnail), drag out two regions on it
(hold down Ctrl and the left mouse button and drag down and right). Your main 
window should look like \fref{fg:twomoreregions}.

\begin{figure}
\figw{3in}{snap12.jpg}
\caption{Two more regions}
\mylabel{fg:twomoreregions}
\end{figure}

\ct{im\_insert} is a VIPS operation that puts one image inside another at an
(x, y) position. VIPS operations work on VIPS images. The \ct{value} member of
an \ct{Image} or \ct{Region} is the VIPS image that underlies the \nip{} row.

You can use \ct{im\_insert} to make a thing to join two images together. Back
in the program window, click on \ctr{File}\ctr{New}\ct{Tool} and enter:

\begin{verbatim}
// join two images left-right

Join a b = class Image value {
  shim = Scale "Spacing" 0 1000 0;
  value = im_insert a.value b.value
    (a.width + shim.value) 0;
}
\end{verbatim}

\noindent
Click \ctr{File}\ct{Process}. This defines a class \ct{Join} which
subclasses the \ct{Image} graphic.  

Now select your two regions (click on the first one, shift-click on
the second) and click on \ctr{Toolkits}\ctr{untitled}\ct{Join}. A
new \ct{Join} row will appear. Open it up and drag the slider to
set the spacing between the two joined images. Go back to the image
viewer for the image file you loaded and try dragging one of the
regions. \fref{fg:myjoin} shows this class in action. The thing in
\ctr{Toolkits}\ctr{Image}\ctr{Join}\ct{Left to Right} is just a fancier
version of this.

\begin{figure}
\figw{3in}{snap13.jpg}
\caption{Joining two images with \ct{Join}}
\mylabel{fg:myjoin}
\end{figure}

You can change how the graphic widgets behave by subclassing them. Try:

\begin{verbatim}
Scale_int c f t v = class
  scope.Scale c f t ((int) v) {
  Scale = Scale_int;
}
\end{verbatim}

\noindent
This defines a new scale class called \ct{Scale\_int} which can only take
integer values. The \ct{Scale = Scale\_int;} line is \ct{Scale\_int}
overriding \ct{Scale}'s constructor, so that a \ct{Scale\_int} stays a
\ct{Scale\_int} when you drag. Because there's a local called \ct{Scale},
\ct{Scale\_int} needs to use \ct{scope.Scale} to refer to the superclass.

Here's a version of \ct{Mark} which can only be dragged in a circle. You pass
it an image to display on, an xy centre position, a radius and a start angle.

\begin{verbatim}
Mark_circle image x y r a = class 
  scope.Mark image _x' _y' {
  // get rect cods for our point
  _pos = (x, y) + rectangular (r, a);
  _x' = re _pos;
  _y' = im _pos;

  Mark i l t 
    = this.Mark_circle i x y r a'
  {
    // vector from centre of
    // circle to new position
    u = (l, t) - (x, y);

    // angle of vector
    a' = im (polar u);
  }
}
\end{verbatim}

