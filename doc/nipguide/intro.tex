\chapter{Getting started}

\nip{} is a user interface for the VIPS image processing library. It is
designed to be fast, even when working with very large images, and to
be easy to extend.  

This guide is split into quite a few chapters:

\begin{itemize}

\item
If you want to use \nip{} to assemble infrared mosaics, you should read
\cref{sec:ir}. The middle section in the tutorial (see \pref{sec:irtut})
does IR mosaics very quickly.

\item
If you want to use \nip{} for general image processing, work through 
\cref{sec:tutorial}.

\item
If you have specific questions about some part of \nip{}'s user-interface, look
at \cref{sec:reference}.

\item
If you're really hardcore, take a look at \cref{sec:program}, which covers
programming.

\item
If you want to know more about VIPS, the image processing package
underlying \nip{}, try the \emph{VIPS Manual}.

\end{itemize}

If \nip{} has installed correctly, you should see something like
\fref{fg:introwin} when it starts up.

\begin{figure}
\figw{3in}{snap1.jpg}
\caption{\nip{} as it starts up}
\label{fg:introwin}
\end{figure}

